\documentclass[10pt,twocolumn,letterpaper]{article}

\usepackage{cvpr}
\usepackage{times}
\usepackage{epsfig}
\usepackage{graphicx}
\usepackage{amsmath}
\usepackage{amssymb}

% Include other packages here, before hyperref.

% If you comment hyperref and then uncomment it, you should delete
% egpaper.aux before re-running latex.  (Or just hit 'q' on the first latex
% run, let it finish, and you should be clear).
\usepackage[pagebackref=true,breaklinks=true,letterpaper=true,colorlinks,bookmarks=false]{hyperref}

\cvprfinalcopy % *** Uncomment this line for the final submission

\def\cvprPaperID{****} % *** Enter the CVPR Paper ID here
\def\httilde{\mbox{\tt\raisebox{-.5ex}{\symbol{126}}}}

% Pages are numbered in submission mode, and unnumbered in camera-ready
\ifcvprfinal\pagestyle{empty}\fi
\begin{document}

%%%%%%%%% TITLE
\title{Deep Learning for Amazon Rainforest Monitoring by Satellite Images}

\author{Rajanie Prabha\\
{\tt\small prabha.rajanie@tum.de}
\and
Chinmay Prabhakar\\
{\tt\small ga53tet@mytum.de}
\and
Sergei Rybakov\\
{\tt\small sergei.rybakov@tum.de}
\and
Min-An Chao\\
{\tt\small minan.chao@tum.de}
}


\maketitle
%\thispagestyle{empty}

%
% Proposal I
%
\section*{Project Proposal}
This project is based on a Kaggle challenge
to automatically label given satellite images in Amazon rainforest 
to monitor the human footprint.

\section{Introduction}
    Deep in the Amazon rainforest where people could hardly approach,
    there are more and more deforestation as well as other illegal activities
    happening without proper administration under the local authorities.
    With the help of satellite images,
    we are able to have entire view over the whole rainforest.
    However without the help of machine learning,
    it is also infeasible to track down every details manually.
    Our goal is to build a neuron network 
    and train it with given dataset with labels,
    so that we will have a model to label the huge space of the rainforest
    automatically.
    These results will be useful to help local authorities to track down
    detailed location of illegal or inappropriate human activities.

    \subsection{Related Works}
        \begin{itemize}
            \item{
                The original Kaggle challenge could be found here:
                https://www.kaggle.com/c/planet-understanding-the-amazon-from-space
            }
            \item{
                There is one related working paper concerning the same topic
                by the Stanford University:
                http://cs231n.stanford.edu/reports/2017/pdfs/915.pdf.
                where several work regarding
                the general processing tricks for computer vision tasks
                using satellite images are also introduced and reviewed.
            }
            \item{
                From the previous results
                we see that the exploration on improving the accuracy
                is not solely on the architecture of network itself,
                but also on defining better loss function
                and on finding an better optimization method.
                So we will probably follow those experiences
                and come up with our own results.
            }
        \end{itemize}

\section{Dataset}
    \begin{itemize}
        \item {
            There are given dataset provided by Planet (a company focusing on environment issue
            and satellite image monitoring) as a standard test data for Kaggle challenge.
            Extension dataset would be also available with other satellite image database
            but we might have to label them ourselves to fit the classes defined by Planet.
        }
        \item {
            The input data are satellite images with larger size.
            Dataset provided by Kaggle are already cropped to a standard size
            in order to make training and testing easier.
            The data for training in this case must contain proper labels,
            and the output will be one text label for each test image.
            However, there would still be some labeling problem
            because in one image there might exist 
            features from two or more classes in the same time.
        }
    \end{itemize}

\section{Methodology}
    As a general architecture,
    we will start by some data augmentation techniques and
    use a pretrained model like VGG or ResNet. 
    We will further fine-tune this model with the satellite images of Amazon
    to get a set of class label probabilities. 
    We will use binary cross entropy loss for calculating the loss function and
    Adam optimizer as a start for our training,
    then we apply the adjustment to conquer the data imbalance problem,
    and finally repeat the whole procedure as shown in related work.

    The further extensions might be the followings.
    (1)~Think more about the data augmentation and data imbalance problem,
    could we also come up with a more generalized way for satellite images?
    We might have to explore more about the related works dealing with 
    satellite images.
    (2)~Since the dataset contains the infra-red dimension,
    we could also think of if we should separate that dimension
    from other RGB channels to make our model learns different things from it.
    (3)~Since one image in this dataset could contain several labels,
    the problem is somehow related to multiple object detection 
    instead of typical classification.
    Maybe we should also work on the methods used for multiple object detection.


\section{Outcome}
    We hope to come up with a baseline accuracy at 90\%
    with the pretrained network as a baseline,
    then we hope to find more domain knowledge to improve our work.
    The satellite image classification would have
    different challenges other than the RGB images we used in our homework.
    We expect to first have a complete repetition after the states of the art,
    and then based on the extensions mentioned above to explore more.
    Those knowledge should be based on some characteristics
    of satellite images or rainforest landscapes
    but not specific rules handcrafted for this project only.

{\small
\bibliographystyle{ieee}
\bibliography{bib}
}

\end{document}
